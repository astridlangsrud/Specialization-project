%===================================== CHAP 2 =================================

\chapter{Neuroscience context}

Before going into the statistics and modeling, it is useful to give some context for the work. The aim of this chapter is to describe the relevant concepts from neuroscience, explain the background for the data material and motivate a practical understanding. Section \ref{structure_nn} gives a brief description of the signaling and connectivity in neural networks, and the hallmarks for Alzheimer's disease in the brain. For the first part the main source used is the book \textit{Neuroscience} \cite{Purves} and for the second part the content is based on sources \cite{Gomez,Witter:2011, Alz}. Section \ref{Lab} provides an outline of the lab experiments where the data material comes from. The information is based on the description of the lab set up in Katrine's master thesis. The same lab set up as will be used for the Alzheimer's experiments. Even though only simulated data will be used in this project work, it is relevant to include this context to make sure that the framework fits to future modeling of the real data. 

%En av de som Katrine sendte
%http://www.scholarpedia.org/article/Entorhinal_cortex
%https://en.wikipedia.org/wiki/Entorhinal_cortex


\section{Concepts from neuroscience}
\label{structure_nn}

\subsection{Neuron and connections}\\
The brain comprise a complex network of neurons that work together by transmitting signals to each other. The basic unit, the neuron, 
 consists of a cell body (soma), an axon and dendrites. Figure \ref{neuron} shows an illustration of a neuron.
\begin{figure}[h]
    \caption{}
    \label{neuron}
    \centering
    \includegraphics[scale=0.3]{Neuron.jpeg}
\end{figure} 

Neurons can transmit electrical signals to each other through a connection of the axon of one neuron with the dendrite of another. The sending and the receiving neuron are referred to as the "presynaptic neuron" and the "postsynaptic neuron", respectively. Such a connection between two neurons is called a synapse, and is in practice a short gap where chemical units, called neurotransmitters, are allowed to flow from the axon to the dendrite. 

This signaling happens in response to something called an action potential in the presynaptic neuron. When a neuron is at rest, there is a constant potential difference between the inside and the outside of the neuron's cell membrane. An action potential is a rapid increase in this membrane potential, caused by ion flows through channels in the membrane. This action potential then propagates along the axon until it reaches the synapse, and ends up as an electrical signal to the connected neuron. In order for the action potential to occur, the potential difference must reach a certain threshold value. If this threshold is reached, the action potential will take place no matter what. In other words there is an all-or-non property, and if it first takes place the action potential will always have the same strength. This phenomenon is illustrated in figure \ref{AP}.

\begin{figure}[h]
\caption{}
\label{AP}
\begin{subfigure}
\centering
\includegraphics[scale=0.7]{AP.jpg}
\end{subfigure}
\begin{subfigure}
\centering
\includegraphics[scale=0.21]{Axjk4.jpg}
\end{subfigure}
\end{figure} 

The developing of an action potential happens in response of some stimuli. This can be external stimuli from some sense, for example a ray of light that hits the eye. Also, action potentials can develop after the receiving of an electrical signal from another neuron. When a signal increase the likelihood of an action potential to also arise in the postsynaptic neuron, this is referred to as an excitatory signal. This enables the possibility for a signal to propagate through the network, and eventually end up for example in a muscle and cause a contraction. On the contrary there are also signals that decrease the chance that the postsynaptic neuron will fire. These are called inhibitory signals.

The  strength of these neural connections are not fixed, but can change over time. Strength of synapses refer to the probability that the postsynaptic neuron will be affected by the signal. A frequent activation of a synapse can strengthen the synaptic connection. This phenomenon is called "long-term potentiation (LTP) of a synapse. Other times activation of a synapse can weaken the connection over time, known as "long-term depression". These changes of connections are referred to as synaptic plasticity, and is the mechanism that gives rise to learning and memory. \\

\subsection{Entorhinal cortex and Alzheimer's disease}

%\begin{itemize}
    %\item More on this section
    %\item Say something about what is known today and what is not. Status of research
    %\item Make connection to the work I am going to do. Why do we think that its an idea to invesigate the learning rule in the context of Alzheimer's research
%\end{itemize}
%The human brain is very good at forgetting. The hippocampus is a structure in the brain that has been associated with various memory functions

The data basis for this project are lab recordings on mice brains with and without AD. These recordings are done on a part of the brain called he entorhinal cortex. This part is used, since the entorhical cortex is where the earliest indications for AD can be seen. Figure \ref{EC} shows where in the brain the entorhinal cortex is located. 


\begin{figure}[h]
    \caption{Illustration of brain showing location of the Entorhinal cortex}
    \label{EC}
    \centering
    \includegraphics[scale=0.35]{Entorhinal_cortex.png}
    \label{brain}
\end{figure} 

The entorhinal cortex is found in the medial temporal lobe and functions as a gateway between the neocortex and the hippocampus. It is a part of the hippocampal memory system (Witter 2011), and is associated with declarative memory and learning. It is commonly subdivided into six layers, I-VI. It seems like the cells in layer II of the entorhinal cortex are affected in the initial stages of the AD. Characteristic to brains with Alzheimer's is loss of synapses and neurons. When a patient have struggled from AD for a long time, the brain changes significantly. It is visually clear that the brain shrinks a lot. Figure \ref{brain1} illustrates how a brain can look like after having AD for many years.


\begin{figure}[h]
\caption{}
    \label{brain1}
    \centering
    \includegraphics[scale=0.5]{brain_slices_alzheimers_0.jpg}

\end{figure} 

Exactly what causes these losses is not known, but it is assumed that the accumulation of some protein types called amyloid plaques and neurofirbillary tangles seen in AD brains, are involved.
It is assumed that changes in the brain begin many years before the patient is diagnosed. Therefore, an important step in the AD research would be to enabling the discovery of AD in the brain when the disease starts to develop.

%The entorhinal cortex (EC) is an area of the brain located in the medial temporal lobe and functions as a hub in a widespread network for memory, navigation and the perception of time.[1] The EC is the main interface between the hippocampus and neocortex.  The entorhinal cortex (EC), in particular (Wikipedia)
%the layer II of the domains located towards the collateral/rhinal fissure, contains neurons that are
%among the very first to undergo pathological alterations associated with the disease. Specifically,
%neurons in this domain develop the initial cortical tau pathology, while layer II is also known to
%exhibit severe neuronal loss already during the pre-clinical stages of AD. Furthermore, layer IIneurons are also subject to early accumulation of intracellular Aβ. In order to help enable the study
%of these early neuronal pathologies the work in this thesis was aimed at developing a platform for
%studying the LEC layer II neuronal population in vitro. (Katrines master)

\section{Data material}

\label{Lab}\\
The data to be used are recordings from mouse brains with and without Alzheimer's. The ?? are currently in a process of developing techniques to keep the neuronal networks alive long enough for the Alzheimer's disease to develop properly??. However, similar experiments have been performed the last years, and the main techniques are the same. 
\begin{itemize}
    \item Several mice some with and some without Alzheimers. Genetically manipulated? 
    \item Brains extracted, slices from Layer 2 taken out. Kept in dishes
    \item Microelectrode arrays. Equally spaced electrodes that can detect action potentials of single neurons
    
\end{itemize}

Each electrode records the electric potential for one neuron over a time interval.  When a neuron undergoes an action potential, this will appear as a peak in these recordings. It is the time points of these action potentials that are of interest, and not the voltage value itself. Let's label the N recorded neurons with numbers $1,2,...,N$. Then, the gathered data is sequences of time points for the action potentials for each neuron
\begin{equation}
    \{\{ap_i\}\}_{i=1}^{N} = \{ap_{i1}, ap_{i2}, ...\}_{i=1}^{N} \quad ap_{ix} \in [0,T]
\end{equation}

where $ap_{ix}$ is the recorded time for the x'th action potential for neuron $i$, and $ap_{ix-1} < ap_{ix}$. Such a sequence of time stamps for neuron firing is called a spike train. (Illustration of spike trains).\\







\cleardoublepage