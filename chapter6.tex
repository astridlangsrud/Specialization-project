%===================================== CHAP 6 =================================

\chapter{Discussion and conclusion}

\begin{itemize}
    \item A few sentences summary of what I am going to present in this chapter
\end{itemize}

\section{Reviewing the results}

\begin{itemize}
    \item Short introduction of the main findings in the result; It is possible to 
\end{itemize}
In short the important points to note from the results is that is possible to infer a time varying synaptic connection from spike data, but with only a smoothness prior the method is dependent on having enough data. 



\section{Further work}

\section{Conclusion}

\begin{itemize}
    \item Short summary of the things that has been done in the project; Neurons form connections to each other, and these connections can vary with time. Various learning rules are hypothesized to govern these dynamics. The strength of connections is reflected by the spiking of neurons. Aim to develop a model to infer time varying connectivity based on spike train data. Spiking is modeled with Bernoulli GLMs. Metropolis-Hastings with smoothness prior used for inference. The method showed to work for enough data material. And as expected it did not work well for little data material per inferred weight. To avoid a solution that jumps around, the prior has to be strict for little data. A strict smoothness prior results in a straight line that cannot learn any weight dynamics. Findings in case 3 suggest that 
\end{itemize}



\cleardoublepage